\documentclass[11pt]{scrartcl}

%%%%%%%%%%%%%%%%%%%%%%%%%%%%%%%%%
%% Ersetzen Sie in den folgenden Zeilen die entsprechenden -Texte-
%% mit den richtigen Werten.
\newcommand{\theNumber}{1.77}
\newcommand{\theGroup}{$\pi$}
\newcommand{\theName}{Kugelkondensator mit Dielektrika Hartgummi und Hartpapier}

\newcommand{\nameM}{Moritz Mackiewicz}
\newcommand{\nameE}{Elias Marquart}
\newcommand{\nameC}{Claus Strasburger}
%%%%%%%%%%%

\usepackage[utf8x]{inputenc}
\usepackage[ngerman]{babel}
\usepackage[T1]{fontenc}
\usepackage{amsmath}

\usepackage[babel,german=quotes]{csquotes}
\usepackage{graphicx}
\usepackage{color}
%\usepackage{here}
\usepackage{listings}
\usepackage{color}
\usepackage{microtype}
\usepackage{tikz}
\usetikzlibrary{shapes}

\setlength{\parindent}{0cm}

\definecolor{dkgreen}{rgb}{0,0.6,0}
\definecolor{gray}{rgb}{0.5,0.5,0.5}
\definecolor{mauve}{rgb}{0.58,0,0.82}
\lstset{ %
  language=Octave,                % the language of the code
  basicstyle=\footnotesize,           % the size of the fonts that are used for the code
  numbers=left,                   % where to put the line-numbers
  numberstyle=\tiny\color{gray},  % the style that is used for the line-numbers
  numbersep=5pt,                  % how far the line-numbers are from the code
  backgroundcolor=\color{white},      % choose the background color. You must add \usepackage{color}
  showspaces=false,               % show spaces adding particular underscores
  showstringspaces=false,         % underline spaces within strings
  showtabs=false,                 % show tabs within strings adding particular underscores
  frame=single,                   % adds a frame around the code
  rulecolor=\color{black},
  tabsize=2,                      % sets default tabsize to 2 spaces
  captionpos=b,                   % sets the caption-position to bottom
  breaklines=true,                % sets automatic line breaking
  breakatwhitespace=false,        % sets if automatic breaks should only happen at whitespace
  title=\lstname,                   % show the filename of files included with \lstinputlisting;
                                  % also try caption instead of title
  %keywordstyle=\color{blue},          % keyword style
  %commentstyle=\color{dkgreen},       % comment style
  %stringstyle=\color{mauve},         % string literal style
  escapeinside={\%*}{*)},            % if you want to add LaTeX within your code
    literate={ö}{{\"o}}1
           {ä}{{\"a}}1
           {ü}{{\"u}}1}

\usepackage{xifthen}

\usepackage{multicol}
\usepackage{paralist}
\usepackage{amsmath}
\usepackage{url}

%Kopf- und Fußzeile
\usepackage{fancyhdr}
\pagestyle{fancy}
\fancyhf{}

%Kopfzeile links bzw. innen
\fancyhead[L]{Praktikum ASP -- Projektaufgabe \theNumber : \theName}
%Kopfzeile rechts bzw. außen
\fancyhead[R]{\thepage}
%Linie oben
\renewcommand{\headrulewidth}{0.5pt}

\renewcommand*\sectfont{\normalcolor\rmfamily\bfseries}
\renewcommand*\descfont{\rmfamily\bfseries}
\setkomafont{dictum}{\normalfont\normalcolor\rmfamily\small}
\renewcommand{\rmdefault}{ppl}

\newcommand{\board}{\textit{BeagleBoard xM} }
\newcommand{\q}[1]{\enquote{#1}}

\newcommand{\beagleIP}{\texttt{192.168.0.1} }
\newcommand{\hostIP}{\texttt{192.168.0.2} }
\newcommand{\putty}{\emph{PuTTY} }
\newcommand{\dsfive}{\emph{DS-5} }


\newcommand{\sheetHeader}[5]{
\begin{center}
\small \textsc{Lehrstuhl f\"ur Rechnertechnik und Rechnerorganisation}\\
\vspace{-.5em}
\Large {\bfseries Aspekte der systemnahen Programmierung\\
\vspace{-.2em}
bei der Spieleentwicklung}\\
\vspace{.5em}
\normalsize #1\\
\vspace{.5em}
#2\\
#3\\
#4\\
#5
\end{center}
}



%\include{__config}
\usepackage{hyperref}

\begin{document}

\sheetHeader{Projektaufgabe \theNumber : \theName}{Gruppe \theGroup}{\nameM}{\nameE}{\nameC}


\section{Problemstellung und Spezifikation}

\subsection{Aufgabenstellung}
In der von uns bearbeiteten Aufgabe geht es um die Berechnung der Kapazität eines Kugelkondensators.

Dazu wird eine Liste von Kondensatorabmessungen in Form einer Textdatei eingegeben. Zu dieser dann in einem Assemblerprogramm die jeweiligen Kapazitäten für verschiedene Dielektrika berechnet werden.
Die Abmessungen bestehen jeweils aus zwei Radien, dem der inneren Kugel $r_{1}$ und dem der äußeren Kugel $r_{2}$.
%% TODO Bild von Kugelkondensator einfügen

\subsection{Implementierungsumgebung}
Wir verwenden zur Entwicklung das \board, ein ARMv8-System mit NEON-VFP. Dieses erlaubt bla, bla und bla %%TODO

Unsere Entwicklungsumgebung besteht aus \emph{ARM Development Studio 5 für Eclipse}, \emph{GNU make} und \emph{gcc 4.7}.

\section{Lösungsalternativen}
\subsection{Unsere Lösung -- Vector Floating Point (VFP)}
\subsubsection{Vorstellung}
In der von uns enwickelten Lösung verwenden die Vector Floating Point Erweierung von ARM. 
Diese erlaubt es uns mit Floats und den Grundrechenarten zu rechnen.
\subsubsection{Vorteile}
\begin{itemize}
\item beliebige Menge an zu verarbeitenden Daten
\end{itemize}
\subsubsection{Nachteile}
\begin{itemize}
\item langsamer als SIMD
\end{itemize}
\subsection{Alternative Lösung -- SIMD (NEON)}
\subsubsection{Vorstellung}
In dieser alternativen Lösung wird der NEON Koprozessor von ARM verwendet, um von SIMD (Single Instruction Multiple Data)
zu profitieren. Das bedeutet, dass pro Assemblerbefehl nicht mit einem, sondern mit vier Floats gerechnet werden. Dadurch würde das Programm schneller werden.
\subsubsection{Vorteile}
\begin{itemize}
\item schnell
\end{itemize}
\subsubsection{Nachteile}
\begin{itemize}
\item SIMD Befehle müssen jeweils 4 Floats verarbeiten
\item Probleme, falls Menge an Floats keine Vielfaches von vier
\end{itemize}
\subsection{Entscheidungsprozess}
%% TODO Entscheidungsprozess erkären
\section{Dokumentation der Implementierung}
\subsection Entwickler-Dokumentation
\subsubsection Rahmenprogramm
\subsubsection Assembleroutine
\subsubsection Optimierungen
\subsection Benutzer-Dokumentation

\section{Ergebnisse}
\subsection Vergleich von Assembler und C-Code
\subsubsection Genauigkeit der Ergebnisse
\subsubsection Laufzeit
\subsection Analyse und Bewertung




\end{document}
